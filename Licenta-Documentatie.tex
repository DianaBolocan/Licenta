\documentclass[12pt,a4paper]{article}

%opening
\title{Generalizarea/Eficentizarea încripției în sisteme criptografice KP-ABE}
\author{Bolocan Diana}

\begin{document}

\maketitle

\begin{abstract}
Se propune o nouă abordare eficientă pentru scheme bazate pe chei secrete si atribute de acces (cunoscut si sub acronimul KP-ABE) pentru circuite Booleene, în care se aduc modificări atât la structura fizică a circuitelor, cât și la sistemul criptografic însoțit. Pornind de la lucrarea [1], se vor prezenta principiile care stau la baza noii scheme și se va demonstra de ce această abordare este mai eficientă decât cea anterioară. 
\end{abstract}

\section{Introducere}
Încripția bazată pe atribute de acces (ABE) este un tip de încripție bazată pe chei publice, în care identitatea utilizatorului este definită ca un set de atribute (exemplu: rolurile unui angajat într-o companie).  

\section{Preliminarii}

\section{Atacul Backtracking}

\section{Noua Construcție}

\section{Complexitate}

\section{Concluzii}

\section{Referințe}
\begin{enumerate}
	\item Ferucio Laurențiu Țiplea și Constantin Cătălin Drăgan. Key-policy Attribute-based Encryption for Boolean Circuits from Bilinear Maps. În BalkanCryptSec 2014, Istanbul, Turcia, 16-17 Octombrie, 2014, paginile 175-193, LNCS 9024
\end{enumerate}

\end{document}
